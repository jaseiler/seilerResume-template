%!TEX TS-program = xelatex
\documentclass[]{seiler-resume}
\addbibresource{bibliography.bib}

\setlength{\topmargin}{-0.3in}
\setlength{\textheight}{10in}
\begin{document}
\header{Jennifer}{Seiler}{PhD}
       {Computational Physicist}
 \begin{textblock}{3.6}(1.7, 1.65)
 \includegraphics[scale=0.5]{gitview2}
 \end{textblock}
% In the aside, each new line forces a line break
%\begin{aside}
%\end{aside}
\begin{emailphone}
\href{https://github.com/jaseiler}{github.com/jaseiler}
\href{https://scholar.google.com/citations?user=I9vG6PAAAAAJ}{Google Scholar}
\href{http://jennseiler.com}{jennseiler.com}
\end{emailphone}
\section{summary}
I am a computational physicist and game developer looking for a stimulating new challenge. For the past 17 years I have worked on various large scale computational modeling and simulation projects. I have also worked in developing and promoting reproducible research and analysis methods, and maintainable shared code. Although my degrees are in physics, I have very strong and unique computer science background in software development and testing, numerical simulations, analysis, database management, and game design.

\section{employment}
\begin{entrylist}
\job{Giant Army}
{Astrophysicist and Developer}{2015 - 2020}
{I work as Developer and Staff Astrophysicist on improving the physics, planetary science, climate simulation, and stellar astronomy of  \href{http://universesandbox.com}{ Universe Sandbox}. The game is a physics-based space simulator that allows users to simulate galaxies, planetary systems, climates, collisions, structure formation, Roche fragmentation, stellar evolution, material phases, liquid and vapor flow, and much more.}
%\bigskip\hspace{.2in}  {{Dan Dixon: }}\href{mailto:dan@universesandbox.com}{dan@universesandbox.com}\\ 
\job{Columbia University}
{Postdoctoral Research Scientist}{2013 - 2015}
{Postdoctoral position in the Department of Statistics researching issues of reproducibility in science. A major focus was \href{http://researchcompendia.science/about/\#AboutUs}{ResearchCompendia.science}. ResearchCompendia is a web service that allows researchers to run codes associated with scientific publications. The service allows authors of publications to create companion websites on which others may reproduce the paper's results or to run their own parameters on virtual machines on AWS servers.}
%\bigskip\hspace{.2in}  {{Dr. Victoria Stodden: }}\href{mailto:vcs@stodden.net}{vcs@stodden.net}\\ 
\job{NASA Goddard SFC}
{Postdoctoral Research Position}{2010 - 2012}
{NASA Postdoctoral Position (NPP) in the numerical relativity group for the \href{https://lisa.nasa.gov/}{LISA project}. I wrote numerical simulations of binary black hole spacetimes, electromagnetic counterparts to black hole interactions, and matter fields around binary black hole systems. I also contributed to the \href{https://einsteintoolkit.org/members.html}{Einstein Toolkit} open source modular physics simulation project for adaptive mesh simulations. }
%\job{Max-Planck Institut f\"{u}r Gravitationsphysik}
%{PhD Research Scholar}{2005 - 2010}
%{PhD work on numerical simulations of black hole spacetimes.  My focus was on well-posed constraint preserving boundary conditions.  With additional work on constraint damping methods, gravitational wave detectability, and phenomenological waveforms and predictions for merged binary final spin and recoil velocity.}
%\job{Albert Einstein Institute}
%{Visiting Scientis}{2004 - 2004}
%{I wrote a parallelised numerical code to generate initial data and evolve a simulation of the propagation of gravitational waves off a potential in a three dimensional coordinate system and track constraint propagation and violation. }
%\job{Cornell University}
%{Undergraduate Researcher}{2002 - 2004}
%{Worked for Prof. Saul Teukolsky on \href{https://www.black-holes.org/code/SpEC.html}{software for the visualization} and analysis of numerical simulations of solutions to the Einstein equations. These included inspiraling neutron stars and black holes systems, binary black holes, and accretion disks.   }
%Called DUSTVis, it is an OpenDX visualization program designed to be used with the Caltech/Cornell DUST algorithm.
%\job{Cornell University}
%{Student Project}{2004 - 2004}
%{Designed software for an industrial chemical waste exchange program, titled the National Trash to Treasure Network, for submission to the EPA as a project for voluntary participation offered to companies as an alternative to fines. }
%A learning algorithm finds other chemicals with similar properties for the buyer and evaluates the cost of transportation and processing. 
%\job{Fermi National Accelerator Lab}
%{Internship}{2002-2002}
%{Participated in the Internship for Physics Majors Program (IPM).  I designed and programmed the track-finding algorithm for the Level 1 Trigger Code for the BTeV project. After finding tracks, it looks for detached tracks which signify an exotic decay, on-the-fly in the detector firmware. }
%\job{Naval Research Laboratories}
%{Internship}{ 2001 - 2001}
%{Worked in the Electronics Science \& Technology Division on the optimization of natural growth of Silicon dioxide, SiGe, and SiC samples via Molecular Beam Epitaxy.  Experimented with the temperature and surface segregation dependencies of Phosphorous doping rates via MBE. }
\end{entrylist}

\section{education}

\begin{entrylist}
\job{Max-Planck-Institut f\"{u}r Gravitationsphysik}{Ph.D., magna cum laude} %
{2005 - 2010}{ {\small \textit{Thesis:} \href{http://jennseiler.com/docs/thesis.pdf}{Numerical Simulations of Binary Black Hole Spacetimes and a Novel Approach to Outer Boundaries}}\\
\textit{Advisers:} Luciano Rezzolla \& Bernard Schutz $\vert$
\textit{University Affiliation:} Leibniz Universit\"{a}t Hannover International Max Plank Research School Fellowship $\vert$ Atoms and Bits volunteer (technology workshops) $\vert$ Journal club organizer $\vert$ IT conference support $\vert$ Lectures at Potsdam University}

\job{Cornell University}{BA Honors Physics}{2001-2005}{\textit{Adviser:} Saul Teukolsky $\vert$
\textit{Research Emphasis:} Computational Physics; \href{https://www.black-holes.org/code/SpEC.html}{Numerical Relativity} \\ \textit{Minors:} Math \& CS $\vert$ Treasurer Society for Physics Students $\vert$ Cornell Parliamentary Debate Team $\vert$ 
Volunteer for Ithaca Big Brothers Big Sisters  $\vert$ Volunteer Planned Parenthood of the Southern Finger Lakes  $\vert$ Sister in Sigma Chi Delta service fraternity  $\vert$ FermiLab Internship | \href{https://www.black-holes.org/code/SpEC.html}{NASA NY Space Grant} }

\job{Hayfield Secondary School}{H.S., Honors A.P.} {1997 - 2001}
{\textit{Research Emphasis:} \href{https://www.washingtonpost.com/archive/local/2000/12/21/honors-and-awards/22049aac-b3e7-4112-8874-317d7f1ed3a0/}{Physics}; Architecture; Computer Science; Electrical Engineering | Poetry Club | Varsity It's Academic | CS 211 with George Mason University | Orchestra (Saxophone/Piano) | Internship at Naval Research Lab in Solid State Physics | Internship at National Superconducting Cyclotron }
\end{entrylist}

\section{skills}
\textbf{\large Software}\\
\textbf{Languages:} C\# $\vert$ C++ $\vert$ Python $\vert$ Java $\vert$ Fortran $\vert$ R $\vert$ Basic $\vert$ Perl $\vert$ Pascal $\vert$ Javascript  $\vert$ HTML/CSS $\vert$ HLSL\ldots\\
\textbf{Frameworks:} Unity $\vert$ TensoFlow $\vert$ numPy $\vert$ Django $\vert$ PANDAS $\vert$ Matplotlib $\vert$ sciPy $\vert$ Hadoop $\vert$ DL4J \ldots \\
%Graphics: Shaders (GLSL) | Generative Geometry and Effects | Particle Systems 
\textbf{Tools:} Git | Photoshop | Matlab | Maple | Octave | Gnuplot | Jira | Subversion | CVS | Emacs | vim \ldots \\
\textbf{Operating Systems:} Windows  | OS X | iOS | Linux (Debian, Ubuntu, RedHat) | Unix (Solaris)\\[1em]
%General: Rapid Idea Prototyping | Detailed Documentation | Reproducible Programming
\begin{tabular*}{\textwidth}{@{\extracolsep{\fill}}ll}
\textbf{\large Mathematics} & \textbf{\large Physics}\\
\parbox[t]{6.6cm}{Statistical Analysis, Finite Element Analysis, N-Body Simulation, Differential Geometry, Discrete Mathematics, Complex System Modeling,
Numerical Integration, Linear Algebra, Vector Calculus, PDEs,  
Dynamic Systems\ldots}& \parbox[t]{7cm}{General Relativity, Climate Modeling, Stellar Evolution, Hydrodynamics, Statistical Mechanics, Lagrangian Mechanics, Acoustics, Optics, Electrical Engineering, Plasma physics, Fusion modeling, Electromagnetism\ldots}
  \end{tabular*}
  
\pagebreak
\section{projects}
\begin{entrylist}
\project{Universe Sandbox}{Astrophysicist \& Developer}{2015-2020}{\small
\href{http://universesandbox.com}{Universe Sandbox} is a physics simulator sandbox game with n-body gravity simulation, EBM and GCM climate modeling, collision physics, roche fragmentation, stellar evolution simulation, and much more. It gives the user the freedom to create and destroy on a celestial scale. I develop much of the simulation code and advise on the science for the entire team. If you are interested in playing the game, please let me know. 
\begin{itemize}[noitemsep, leftmargin=0.15in, parsep=0pt, partopsep=0pt]
	\item I wrote game code in {\bf C\#} for {\bf Unity}. I helped with the visuals as well as the data simulation side of the code.
	\item I wrote grid-based simulation code to run on the GPU with {\bf HLSL} Unity shaders. Some of those shader included climate simulation, heat diffusion simulation, heating and height-map effects from collisions, water and vapor flow, and phase tracking for potential atmospheric materials.
	\item I wrote stellar evolution code including: a method to identify initial stellar properties based on the age and mass of the star; a method to evolve the stars that solves simultaneously for the structure and chemical composition of the star based on the calculated material burning rates for the core and envelope of the initial star; code to fit to fixed mesh point along selected stellar evolutionary tracks, determine type transitions, and deal with collapse, collisions and fragmentation. I wrote scripts for to calculate parameter values and mesh points using {\bf Python}.
	\item I wrote systems for material volatilization due to atmospheric escape, erosion, and collisions.
	\item I work remotely. We track tasks using {\bf Trello} and {\bf Zenhub}, use {\bf Git} for VC, and {\bf Slack} for communication.
\end{itemize}
}

\project{ResearchCompendia.science}{Developer}{2013-2015}{ \small
The \href{http://researchcompendia.science/about/\#AboutUs}{ResearchCompendia} platform is an attempt to use the web to enhance the reproducibility and verifiability of scientific research. We provide the tools to publish the "actual scholarship" by hosting data, code, and methods in a form that is accessible, trackable, and persistent. 
\begin{itemize}[noitemsep, leftmargin=0.15in, parsep=0pt, partopsep=0pt]
	\item I worked on the frontend and backend of the site in {\bf Django}, {\bf Python}, {\bf json} and {\bf SQL} to provide for user account creation, data and meta-data storage, and
	remote executability of contributed code for a variety of coding languages and frameworks via virtual machines on our {\bf AWS} servers using {\bf Docker}.
	\item I wrote {\bf Python} scripts to provide DataCite DOI issuing for compendia data and codes, and uploaded code and data objects, and allow those identifiers to link to the original
	publications.
	\item I created customized portals for journals and institutions requiring differential access and tracking, and scripted the automated creation of hundreds of compendia.
	\item The site source is an entirely open source platform written in Django (python) and \href{https://github.com/researchcompendia/researchcompendia}{available on github}.
	We wrote documentation of the project and for potential contributors on \href{https://researchcompendia.readthedocs.io/}{Readthedocs}.
	\item I used {\bf travis CI} for continuous integration testing, {\bf PostgreSQL} for database management, {\bf Celery} for asynchronous task management, and {\bf Heroku} for deployment.
%	\item We provide tools to share and archive the data, codes, documentation, parameters, and environmental settings linked with published research all in one place. We support the verification and validation processes by providing for the remote execution of codes in our cloud resources. We make these tools easy to utilize to lessen the exertion required from overburdened researchers in the process of publishing fully reproducible work.
\end{itemize}}
\project{Analysis of Computational Reproducibility}{Statistical Researcher}{2013-2015}{\small
For \href{https://www.pnas.org/content/115/11/2584}{ a paper} titled \textbf{An Empirical Analysis of Journal Policy Effectiveness for Computational Reproducibility} I reproduced the portions of statistical and computational results of most of 300 papers published in Science for which we could obtain the necessary data and code. Data and code was obtained either via request or supplementary material. 
\begin{itemize}[noitemsep, leftmargin=0.15in, parsep=0pt, partopsep=0pt]
	\item I analyzed all articles for the relevant series of issues of Science since the open data/code policy was established and determined which could have computational reproducible results and or analysis. I then established which provided enough via journal supplementary materials or cited online locations, and which needed requests to  the authors.
	\item I tracked and categorized all papers with regard to extent of data and code provided, responses to requests, and reproducibility conditions.
	\item  I reproduced statistical results using the data provided using {\bf R, Python, Mathematica, and Maple}, I reproduced computational results codes provided and (where possible) judged the code itself if it was in {\bf C++, C, Fortran, Python, or Java}.
\end{itemize}}
%\job{Co-Author on Three of the Top Cited GR Papers of 2009}{Astrophysicist}{2006-2009}{
%The three were: ?Testing gravitational-wave (GW) searches with numerical relativity waveforms: Results from the first Numerical INJection Analysis (NINJA) project?, ?On the final spin from the coalescence of two black holes?, and ?The Final spin from the coalescence of aligned-spin black-hole binaries?.  I have a very high citation rate overall, but those also represent a fair sampling of my research progress.  I entered the field of numerical relativity two years before the first successful binary black hole simulation in 2005.  Since then I have been involved in some of the major discoveries of the field:
%
%Developed a phenomenological formulae to predict the final spin, recoil, and orientation of a merged black hole given only initial data from the originating binary system.
%We worked with data analysts with the LIGO detector in the NINJA project to test how successful our detector pipelines would be at identifying signals matched against the database of our numerical waveforms.
%We matched post-Newtonian waveforms for binaries with large separations to numerically generated waveforms for close in binaries?where only numerical simulations can deal with the non-linearity of the system?to produce very long and accurate waveforms for GW detection.}
\project{Numerical Simulation of Black Hole Spacetimes}{Astrophysicist}{2002-2012}{\small
I worked extensively on super-computer simulations of relativistic space-times, and on a framework for numerical simulation called \href{https://einsteintoolkit.org/members.html}{the Einstein Toolkit}. Gravitational Wave (GW) detectors need numerical GW templates for signal recognition by detector pipelines, and binary black hole inspirals are the strongest source for GW signals. My research focused on improving simulations, and on the generation of numerical GWs both for detector templates and for astrophysical predictions.
\begin{itemize}[noitemsep, leftmargin=0.15in, parsep=0pt, partopsep=0pt]
	\item I developed and tested massively parallel numerical simulations that evolve highly non-linear partial differential equations (the Einstein Equations) for close binary black holes as part of a large scale collaborative ``toolkit'' written in {\bf C++/Fortran/Python}.
	\item I helped develop multiple "thorns" for \href{http://cactuscode.org}{Cactus} and \href{https://einsteintoolkit.org/members.html}{the Einstein toolkit}, including: the AEIHarmonic evolution code for evolving full 3D evolutions of the Einstein Equations; two well-posed, constraint preserving boundary conditions thorns for two different coordinate systems; Teukolsky Wave initial data thorn; multiple visualization thorns; many analysis thorns, including horizon mass estimation, and spin orientation analysis
	\item I contributed to the adaptive mesh refinement project, Carpet, {\bf HDF5} I/O for  {\bf Amira, VisIt} and {\bf OpenDX}, memory allocation, parallelization with {\bf MPI} and {\bf OpenMP}, scalability, overall {\bf C++} optimization, and cross architecture compatibility
	\item  For analysis I use {\bf Python, R, Gnuplot, Octave, Matlab, Mathematica}, or {\bf Maple}. The scripting for simulation management and analysis I usually use {\bf Python, Ruby, bash}, or {\bf Perl}. For versioning we used {\bf svn}, then {\bf darcs}, then {\bf git}.
\end{itemize}}

\end{entrylist}\pagebreak

\section{publications}
{ {L. Rezzolla, P. Diener, E. N. Dorband, D. Pollney, C. Reisswig, E. Schnetter, {\bf J. Seiler}.}  \textbf{The Final Spin From the Coalescence of Aligned-spin Black-hole Binaries}.  \textit{Astrophys.\ J.\  {\bf 674} (2008) L29}.  
Preprint:   \href{http://arxiv.org/abs/0710.3345}{arXiv.org:0710.3345} [gr-qc]}\vspace{0.2cm} \\*
{ {L.~Rezzolla, E.~Barausse, E.~N.~Dorband, D.~Pollney, C.~Reisswig, {\bf J.~Seiler} and S.~Husa}. \textbf{On the final spin from the coalescence of two black holes}.  \textit{Phys.\ Rev.\  D {\bf 78} (2008) 044002}. \\ Preprint: \href{http://arxiv.org/abs/0712.3541}{arXiv:0712.3541}[gr-qc] \vspace{0.2cm} \\
{{\bf J.~Seiler}, B.~Szilagyi, D.~Pollney}.  \textbf{Constraint Preserving Boundaries for a Generalized Harmonic Evolution Systems}.  \textit{Class.\ Quant.\ Grav.\  {\bf 25} (2008) 175020}. Preprint: \href{http://arxiv.org/abs/0802.3341}{arXiv:0802.3341} [gr-qc]\vspace{0.2cm} \\*
{B.~Aylott, {\it et al.}~(including {\bf J.~Seiler})}. \textbf{Testing gravitational-wave searches with numerical relativity waveforms:
Results from the first Numerical INJection Analysis (NINJA) project}.
\textit{Class. \ Quant. \ Grav. \ {\bf 26} (2009) 165008}. Preprint: \href{http://arxiv.org/abs/0901.4399}{arXiv:0901.4399} [gr-qc]. \vspace{0.2cm} \\
{B.~Aylott, {\it et al.}(including {\bf J.~Seiler})}. \textbf{Status of NINJA: the Numerical INJection Analysis project}.  \textit{Class.\ Quant.\ Grav.\ {\bf 26} (2009) 114008}. Preprint: \href{http://arxiv.org/abs/0905.4227}{arXiv:0905.4227} [gr-qc] \vspace{0.2cm} \\
{C.~Reisswig, S.~Husa, L.~Rezzolla, E.~Dorband, D.~Pollney and {\bf J.~Seiler}}.
\textbf{Gravitational-wave detectability of equal-mass black-hole binaries with aligned spins}. \textit{Phys.\ Rev.\  D {\bf 80} (2009) 124026}. Preprint: \href{http://arxiv.org/abs/0907.0462}{arXiv:0907.0462} [gr-qc] \vspace{0.2cm} \\*
{L.~Santamaria, F.~Ohme, P.~Ajith, B.~Bruegmann, N.~Dorband, M.~Hannam, S.~Husa, P.~Moesta, D.~Pollney, C.~Reisswig, E.~L.~Robinson, {\bf J.~Seiler}, B. Krishnan. }\textbf{Matching post-Newtonian and numerical relativity waveforms: systematic errors and a new phenomenological model for non-precessing black hole binaries}
 \textit{Phys.\ Rev.\  D {\bf 82} (2010) 064016}.  Preprint: \href{http://arxiv.org/abs/1005.3306}{arXiv:1005.3306} [gr-qc]\vspace{0.2cm} \\*
{P.~Ajith, M.~Hannam, S.~Husa, Y.~Chen, B. Bruegmann, N. Dorband, D. Muller, F. Ohme, D. Pollney, C. Reisswig, L. Santamaria, {\bf J.~Seiler}.} \textbf{``Complete'' gravitational-waveforms for black-hole binaries with non-precessing spins}.  \textit{Phys. Rev. Lett. {\bf 106} (2011) 241101} Preprint: \href{http://arxiv.org/abs/0909.2867}{arXiv:0909.2867} [gr-qc]\vspace{0.2cm} \\*
{V.~Stodden, S.~Miguez, {\bf J.~Seiler}.} \textbf{ResearchCompendia.org: Cyberinfrastructure for Reproducibility and Collaboration in Computational Science}. \textit{IEEE Computing in Science \& Engineering {\bf 17(1)} (2015) 12-19}. Access: \href{http://online.qmags.com/CISE0115?pg=14&mode=2#pg14&mode2?fs=2&pg=14&mode=2}{Scientific Software Communities}\vspace{0.2cm} \\*
{V.~Stodden, {\bf J.~Seiler}, Z.~Ma. }\textbf{An empirical analysis of journal policy effectiveness for computational reproducibility}. \textit{Proceedings of the National Academy of Sciences {\bf 115.11} (2018): \href{https://www.pnas.org/content/115/11/2584}{2584-2589}.}}%\vspace{-.25cm} }

\section{honors \& awards}
\begin{tabbing}
{\textbullet~  Visiting Scientist Grant from Universitat de les Illes Balears for October 2010}
\\{\textbullet~ \href{http://www.slac.stanford.edu/spires/topcites/2009/eprints/to_gr-qc_annual.shtml}{Three of the top cited GR papers of 2009}}
\\{\textbullet~ \href{http://arxiv.org/pdf/0709.0942}{James Hartle Award}:} \textit{Constraint Preserving Boundaries in 2nd Order Form} at GRG18
%\\{\href{http://arxiv.org/pdf/0709.0942}{\hspace{.2in}\textit{http://arxiv.org/pdf/0709.0942}}}
\\{\textbullet~ NASA NY Space Grant 2003:} For work under Saul Teukolsky on DUSTVis
\\{\textbullet~ FermiLab Internships for Physics Majors 2002:} BTeV Trigger Algorithm
\\{\textbullet~ US DoD's SEAP (Science and Engineering Apprenticeship Program) 2001}
\\{\textbullet~} \={Award of Recognition of Outstanding Achievement 2001, Solid State Physics}
\\ \>{from the US Naval Research Laboratory}
\\{\textbullet~ Treasurer for the Cornell Chapter of Society of Physics Students 2003-2005}
\\{\textbullet~ Recognition from Nat. Science Teachers Association 2000}
\\{\textbullet~ Recognition from Graduate Women in Science 2000, Theoretical Physics }
\\ \>{(for \textit{Acoustic Thermometry of Sea Water}) }
\\{\textbullet~ Intel Science Talent Search Semifinalist 2001 (for \textit{Longitudinal Flow in Au-Au Collisions})} 
%\\{\href{http://www.sciserv.org/sts/60sts/semi_VA.asp}{\hspace{.2in}\textit{http://www.sciserv.org/sts/60sts/semi\_{VA.asp}}}}
\\{\textbullet~ University of Southern California Young Scientist of the Year 2000 }
\\{\textbullet~ CIA Outstanding Young Scientist (for \textit{Acoustic Thermometry of Sea Water}) }
\\{\textbullet~ Armed Forces and Communications and Electronics scholarship summer of '99 }
\\{\textbullet~ Award of Recognition: Society of Women Engineers (\textit{Acoustic Thermometry of Sea Water}) }
%\\{Intel Virginia State Science Talent Search 2nd place (for \textit{Longitudinal Flow in Au-Au Collisions})}
%\\{Grand Prize in the 1999 NOVA Intel Science and Engineering Fair}
%\\{Honors Science Program at Michigan State University in the National Superconducting Cyclotron}
\\{\textbullet~ Physlink.com Young Scientist of the Year 2000 (for Acoustic Thermometry of Sea Water)}
\end{tabbing}
%\\{\href{http://web.archive.org/web/20000816230028/www.physlink.com/ysaward2000_press.cfm}{\hspace{.2in}\textit{http://www.physlink.com/ysaward2000\_{press.cfm}}}}}
\section{extracurricular} {Coursera/EdX/Udacity classes, Physics outreach, Fire performance (poi, rope dart, staff), Open source programming, Electronics, Arduino,
Sustainability outreach, Indoor and outdoor rock climbing, Skiing, Hiking, Kayaking, Scuba, Go, Interactive multimedia installation art, Burning Man community, Vegetarian cooking} 
\end{document}